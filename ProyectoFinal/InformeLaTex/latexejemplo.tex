\documentclass{article}
\usepackage{makeidx}
\makeindex
\usepackage[top=3cm, bottom=3cm, outer=3cm, inner=3cm]{geometry}
\usepackage{multicol}
\usepackage[utf8]{inputenc}
\usepackage{graphicx}
\usepackage{url}
\usepackage{hyperref}
\usepackage{array}
\usepackage{natbib}
\usepackage{pdfpages}
\usepackage{multirow}
\usepackage[normalem]{ulem}
\useunder{\uline}{\ul}{}
\usepackage{svg}
\usepackage{xcolor}
\usepackage{listings}
\lstdefinestyle{ascii-tree}{
    literate={├}{|}1 {─}{--}1 {└}{+}1 
}
\lstset{basicstyle=\ttfamily,
  showstringspaces=false,
  commentstyle=\color{red},
  keywordstyle=\color{blue}
}
\lstdefinelanguage{pseudocode}{
  morekeywords={Inicio, Leer, Como, Entero, Sin, Signo, sum, Para, desde, Hasta, Hacer, Si, Módulo, es, igual, a, Entonces, Fin, Para, Sino, Escribir},
  sensitive=false,
  morecomment=[l]{\#},
  morestring=[b]",
  moredelim=[s][\color{blue}]{\{}{\}},
  moredelim=[s][\color{red}]{\{}{\}}
}

\lstset{
  language=pseudocode,
  basicstyle=\ttfamily,
  keywordstyle=\color{blue},
  commentstyle=\color{red},
  stringstyle=\color{green},
  numbers=left,
  numberstyle=\tiny,
  stepnumber=1,
  numbersep=5pt,
  backgroundcolor=\color{white},
  showspaces=false,
  showstringspaces=false,
  showtabs=false,
  frame=single,
  rulecolor=\color{black}
}
\usepackage{caption}
\usepackage{subcaption}
\usepackage{float}

\newcolumntype{x}[1]{>{\centering\arraybackslash\hspace{0pt}}p{#1}}
\newcolumntype{M}[1]{>{\centering\arraybackslash}m{#1}}
\newcolumntype{N}{@{}m{0pt}@{}}

\newcommand{\itemCourse}{Lenguaje de Programación}
\newcommand{\itemCourseCode}{20231001}
\newcommand{\itemSemester}{IV }
\newcommand{\itemUniversity}{Universidad La Salle}
\newcommand{\itemFaculty}{Facultad de Ingenierías}
\newcommand{\itemDepartment}{Departamento de Ingeniería y Matemáticas}
\newcommand{\itemSchool}{Carrera Profesional de Ingeniería de Software}
\newcommand{\itemAcademic}{2024 - B}
\newcommand{\itemInput}{Del 21 Agosto 2023}
\newcommand{\itemOutput}{Al 26 Agosto 2023}
\newcommand{\itemPracticeNumber}{01}
\newcommand{\itemTheme}{Git y GitHub}

\usepackage[english,spanish]{babel}
\usepackage[utf8]{inputenc}
\AtBeginDocument{\selectlanguage{spanish}}
\renewcommand{\figurename}{Figura}
\renewcommand{\refname}{Referencias}
\renewcommand{\tablename}{Tabla}
\AtBeginDocument{\renewcommand\tablename{Tabla}}

\usepackage{fancyhdr}
\pagestyle{fancy}
\fancyhf{}
\setlength{\headheight}{40.78929pt} % Ajustado según la advertencia
\addtolength{\topmargin}{-0.78929pt} % Ajustado para evitar advertencia
\renewcommand{\headrulewidth}{1pt}
\renewcommand{\footrulewidth}{1pt}
\fancyhead[L]{\raisebox{-0.2\height}{\includegraphics[width=3cm]{img/logo_salle.png}}}
\fancyhead[C]{\fontsize{7}{7}\selectfont \itemUniversity \\ \itemFaculty \\ \itemDepartment \\ \itemSchool \\ \textbf{\itemCourse}}
\fancyfoot[L]{Grupo The Unity}
\fancyfoot[C]{Página \thepage}
\fancyfoot[R]{\itemCourse}

\usepackage{color, colortbl}
\definecolor{dkgreen}{rgb}{0,0.6,0}
\definecolor{gray}{rgb}{0.5,0.5,0.5}
\definecolor{mauve}{rgb}{0.58,0,0.82}
\definecolor{codebackground}{rgb}{0.95, 0.95, 0.92}
\definecolor{tablebackground}{rgb}{0.0, 0.45, 0.63}

\lstset{frame=tb,
	language=bash,
	aboveskip=3mm,
	belowskip=3mm,
	showstringspaces=false,
	columns=flexible,
	basicstyle={\small\ttfamily},
	numbers=none,
	numberstyle=\tiny\color{gray},
	keywordstyle=\color{blue},
	commentstyle=\color{dkgreen},
	stringstyle=\color{mauve},
	breaklines=true,
	breakatwhitespace=true,
	tabsize=3,
	backgroundcolor= \color{codebackground},
}

\begin{document}

    \printindexÍndice de contenido
    
	\vspace*{10px}

	\begin{center}	
		\fontsize{17}{17} \textbf{ Informe de Laboratorio \itemPracticeNumber}
	\end{center}
	\centerline{\textbf{\Large Tema: \itemTheme}}

	\begin{flushright}
		\begin{tabular}{|M{2.5cm}|N|}
			\hline 
			\rowcolor{tablebackground}
			\color{white} \textbf{Nota}  \\
			\hline 
			     \\[30pt]
			\hline 			
		\end{tabular}
	\end{flushright}	

	\begin{table}[H]
    \centering
    \begin{tabular}{|x{5.5cm}|x{4cm}|x{4.8cm}|}
        \hline 
        \rowcolor{tablebackground}
        \color{white} \textbf{Integrantes} & 
        \color{white} \textbf{Escuela}  & 
        \color{white} \textbf{Asignatura}   \\
        \hline 
        Joshua David Ortiz Rosas & \multirow{10}{=}{\itemSchool} & \multirow{10}{=}{\itemCourse \par Semestre: \itemSemester \par Código: \itemCourseCode} \\
        jortizr@ulasalle.edu.pe & & \\
        Miguel Angel Flores Leon & & \\
        mfloresl@ulasalle.edu.pe & & \\
        Abimael Ernesto Frontado Fajardo & & \\
        afrontadof@ulasalle.edu.pe & & \\
        Christofer Alberto Gamio Huaman & & \\
        cgamioh@ulasalle.edu.pe & & \\
        Frederick Dicarlo Mares Graos & & \\
        fmaresg@ulasalle.edu.pe & & \\
        \hline 
    \end{tabular}
\end{table}


	\vspace{0.3cm}
	\noindent
	La entrega del informe es para el curso de Lenguaje de Programación de la Universidad La Salle del semestre \itemSemester del presente año \itemAcademic. La entrega de la práctica es para el número \itemPracticeNumber, que corresponde al tema: \itemTheme.

	\vspace{0.5cm}
	\section{Enfoque del Software}
    \subsection{Objetivo}
	En este informe se muestra el procedimiento y resultados obtenidos durante la práctica número \itemPracticeNumber, en el cual se abordaron temas relacionados con \itemTheme. Se emplearon herramientas como Git y GitHub para el manejo de versiones del código y su maduración, desarrollo grupal y colaboración en nuestro proyecto de programación. Nuestro proyecto está en la busqueda de los 20 o más primeros números perfectos en un corto periodo de tiempo.
	\subsection{Equipos, materiales y conceptos utilizados}
    \begin{itemize}
        \item Sistema Operativo (Windows o GNU/Linux).
        \item Editor de texto plano (Visual Estudio Code o Vim).
        \item Web OmegaUp
        \item Aplicación Git.
        \item Cuenta en GitHub asociada al correo institucional.
        \item Teoría sobre los números perfectos.
        \item Números primos de Mersenne.
        \item Inteligencia Artificial (Meta AI o Chat GPT)
    \end{itemize}
    \section{Marco Teórico}
    \subsection{¿Qué son los Números Perfectos?}
    Los números perfectos son aquellos números enteros positivos [Z+] que son iguales a la suma de sus divisores propios(excluyendo al propio número). Estos divisores, también pertenecen al conjunto de los enteros positivos. Por ejemplo, el número 6 es perfecto porque sus divisores propios son 1, 2 y 3, y 1 + 2 + 3 = 6. 
    \subsection{Números Perfectos en Omega Up}
    \section{Resolución del Problema}
\subsection{Pseudocódigo Uno}
\lstset{language=pseudocode}
\begin{lstlisting}
Inicio
    Leer X Como Entero Sin Signo
    sum = 0
    Para i desde 1 Hasta X / 2 Hacer
        Si X Módulo i es igual a 0 Entonces
            sum = sum + i    
        Fin Si
    Fin Para
    Si sum es igual a X Entonces
        Imprimir "si"
    Sino
        Imprimir "no"
    Fin Si
Fin
\end{lstlisting}

\subsection{Pseudocódigo Dos}
\lstset{language=pseudocode}
\begin{lstlisting}
Inicio
    Leer X Como Entero Sin Signo
    sum = 0
    Para i desde 1 Hasta X / 2 Hacer
        Si X Módulo i es igual a 0 Entonces
            sum = sum + i  
        Fin Si
    Fin Para
    Si sum es igual a X Entonces
        Imprimir "si"
    Sino
        Imprimir "no"
    Fin Si
Fin
\end{lstlisting}

\subsection{Pseudocódigo Tres}
\lstset{language=pseudocode}
\begin{lstlisting}
Inicio
    Leer X Como Entero Sin Signo (ulong)
    sum = 0
    Para i desde 1 hasta la raíz cuadrada de X Hacer
        Si X Módulo i es igual a 0 Entonces
            sum = sum + i
            Si i es distinto de 1 y i es distinto de X dividido por i Entonces
                sum = sum + (X dividido por i)
            Fin Si
        Fin Si
    Fin Para
    Si sum es igual a X Entonces
        Escribir "SI"
    Sino
        Escribir "NO"
    Fin Si
Fin
\end{lstlisting}

\subsection{Pseudocódigo Cuatro}
\lstset{language=pseudocode}
\begin{lstlisting}
Inicio
    Escribir "Ingrese un número para calcular si es perfecto:"
    Leer X Como Entero Sin Signo (ulong)
    sum = 0
    Para i desde 1 Hasta la raíz cuadrada de X Hacer
        Si X Módulo i es igual a 0 Entonces
            sum = sum + i
            Si i es distinto de 1 y i es distinto de X dividido por i Entonces
                sum = sum + (X dividido por i)
            Fin Si
        Fin Si
    Fin Para
    Si sum es igual a X Entonces
        Escribir "SI"
    Sino
        Escribir "NO"
    Fin Si
Fin
\end{lstlisting}

    \section{Uso de Git y Github}
    \subsection{Creación de Repositorio}
	\begin{lstlisting}[language=bash,caption={Creando directorio de trabajo}]
	$ mkdir -p $HOME/code/
	\end{lstlisting}
	\begin{lstlisting}[language=bash,caption={Dirigiéndonos al directorio de trabajo}]
	$ cd $HOME/code/
	\end{lstlisting}
	\begin{lstlisting}[language=bash,caption={Creando directorio para repositorio GitHub}]
	$ mkdir -p $HOME/afrontado/lp3
	\end{lstlisting}
	\begin{lstlisting}[language=bash,caption={Inicializando directorio para repositorio GitHub}]
	$ cd $HOME/afrontado/lp3
	$ git init
	$ git config --global user.name "Abimael Frontado Fajardo"
	$ git config --global user.email afrontado@ulasalle.edu.pe
	$ git add README.md
	$ git commit -m "first commit"
	$ git branch -M main
	$ git remote add origin https://github.com/jperez/lp3-23b.git
	$ git push -u origin main
	\end{lstlisting}
    \vspace{0.5cm}
	\subsection{URL del Repositorio}
	   \begin{itemize}
        \item URL de nuestro Repositorio GitHub
        \item https://github.com/AbimaelFrontado/lp3
      \end{itemize}
	\vspace{0.5cm}
    \subsection{Uso de commits}
    Ahora que estamos más familiarizados con Git y GitHub. Empezaremos con el uso de commits para versionar nuestro código y poder apreciar su propio proceso de maduración
    
	\section{Actividades con el Repositorio}
	En esta práctica se logró la correcta configuración de un repositorio GitHub, así como la comprensión de los comandos básicos para su manejo. El uso de herramientas como Git y GitHub es esencial para el trabajo colaborativo y la gestión de versiones en proyectos de software.
    
	\clearpage
	
	
    \begin{lstlisting}[language=bash,caption={Código}][H]
        using System;

        class Program
        {
           static void Main()
            {
                Console.WriteLine("Ingrese un número para calcular si es perfecto:");
                
                // Leer el número de entrada
                ulong X = ulong.Parse(Console.ReadLine());
        
                // Calcular la suma de los divisores del número (excepto el mismo)
                ulong sum = 0;
                
                for (ulong i = 1; i <= (ulong)Math.Sqrt(X); i++)
                {
                    if (X % i == 0)
                    {
                        sum += i;
                        if (i != 1 && i != X / i) // Evita sumar X y duplicar i cuando es el mismo divisor
                        {
                            sum += X / i;
                        }
                    }
                }
                // Verificar si la suma de los divisores es igual al número
                if (sum == X)
                {
                    Console.WriteLine("SI");
                }
                else
                {
                    Console.WriteLine("NO");
                }
            }
        }
    \end{lstlisting}
	\section{Compilación en Omega Up}
	\subsection {\raisebox{-0.9\height}{\includegraphics[width=14cm]{img/Omegaup.png}}}
	\begin{lstlisting}[language=bash,caption={Commit: Agregar código de números perfectos OMEGAUP}][H]
		$ git add .
		$ git commit -m "Agregar código de números perfectos OMEGAUP"			
		$ git push -u origin main
	\end{lstlisting}

	\subsection{Estructura de laboratorio 01}
	\begin{itemize}	
		\item El contenido que se entrega en este laboratorio es el siguiente:
	\end{itemize}
	
\begin{lstlisting}[style=ascii-tree]
.
|--- 01-lab
|   	|--- 00_prueba.cs
|   	|--- 01_prueba.cs
|		|--- 02_prueba.cs
|		|--- programa.cs
|				|--- HolaMundo.java
|				|--- Test.java
|				|--- Main.java
|		|--- latex [DIRECTORY]
|				|---informe.tex

3 directories, 7 files

\end{lstlisting}    
		
\section{Pregunta:}

¿Cómo podríamos saber con exactitud cuál de todas las propuestas algorítmicas es la más óptima? sugiera y explique a nivel conceptual:

Primeramente se debe de priorizar la efectividad del programa, luego el tiempo de ejecución y finalmente la memoria utilizada para identificar cuál de todas las propuestas algorítmicas sería la más óptima. Debe de ser efectivo, tener el menor tiempo de ejecución posible, así como la memoria utilizada debe ser la menor posible.\\
\section{\textcolor{red}{Calificación}}

\begin{table}[H]
\caption{Rúbrica para contenido del Informe y evidencias}
\setlength{\tabcolsep}{0.5em} % for the horizontal padding
{\renewcommand{\arraystretch}{1.5}% for the vertical padding
%\begin{center}
\begin{tabular}{|p{2.7cm}|p{7cm}|x{1.3cm}|p{1.2cm}|p{1.5cm}|p{1.1cm}|}
	\hline
	\multicolumn{2}{|c|}{Contenido y demostración} & Puntos & Checklist & Estudiante & Profesor\\
	\hline
	\textbf{1. GitHub} & Repositorio se pudo clonar y se evidencia la estructura adecuada para revisar los entregables. (Se descontará puntos por error o observación) &4 &X &4 & \\ 
	\hline
	\textbf{2. Commits} &  Hay porciones de código fuente asociado a los commits planificados con explicaciones detalladas. (El profesor puede preguntar para refrendar calificación). &4 &X &4 & \\ 
	\hline 
	\textbf{3. Ejecución} & Se incluyen comandos para ejecuciones y pruebas del código fuente explicadas gradualmente que permitirían replicar el proyecto. (Se descontará puntos por cada omisión) &4 &X &4 & \\ 
	\hline			
	\textbf{4. Pregunta} & Se responde con completitud a la pregunta formulada en la tarea.  (El profesor puede preguntar para refrendar calificación).  &2 &X &2 & \\ 
	\hline				
	\textbf{7. Ortografía} & El documento no muestra errores ortográficos. (Se descontará puntos por error encontrado) &2 &X &2 & \\ 
	\hline 
	\textcolor{red}{\textbf{8. Madurez}} & \textcolor{blue}{El Informe muestra de manera general una evolución de la madurez del código fuente con explicaciones puntuales pero precisas, agregando diagramas generados a partir del código fuente y refleja un acabado impecable. (El profesor puede preguntar para refrendar calificación).}  &4 &X &4 & \\ 
	\hline
	\multicolumn{2}{|c|}{\textbf{Total}} &20 & &16 & \\ 
	\hline
\end{tabular}
%\end{center}
%\label{tab:multicol}
}
\end{table}
	
	
	
\clearpage

 
\section{Referencias}
\begin{itemize}			
	\item \url{}
	\item \url{}
\end{itemize}	
	
%\clearpage
%\bibliographystyle{apalike}
%\bibliographystyle{IEEEtranN}
%\bibliography{bibliography}
			
\end{document}